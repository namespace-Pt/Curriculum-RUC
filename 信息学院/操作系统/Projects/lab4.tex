\documentclass[UTF8]{ctexart}
\usepackage{dirtree}
\usepackage{listings}
\usepackage{xcolor}
\usepackage{graphicx}
\usepackage{enumerate}
\usepackage[a4paper]{geometry} 
\usepackage{amsmath,amsthm,mathtools}
\usepackage{mathtools}
\usepackage{diagbox}
\usepackage{multirow,makecell}
\usepackage{url}
\newcommand{\refe}[1]{Eq.\ref{#1}}
\newtheorem{theory}{Theory}[section]
\newcommand{\reffig}[1]{\,\emph{图\ref{#1}}\,}
\title{Lab4-进程通讯}
\author{张配天-2018202180}
\date{\today}
\geometry{head=1cm,bottom=1cm,left=2cm,right=2cm}

\lstset{
 columns=fixed,       
 numbers=left,                                        % 在左侧显示行号
 numberstyle=\tiny\color{gray},                       % 设定行号格式
 basicstyle=\ttfamily,
 frame=none,                                          % 不显示背景边框
 backgroundcolor=\color[RGB]{245,245,244},            % 设定背景颜色
 keywordstyle=\color[RGB]{40,40,255},                 % 设定关键字颜色
 numberstyle=\footnotesize\color{darkgray},           
 commentstyle=\color{gray}\ttfamily,                % 设置代码注释的格式
 stringstyle=\rmfamily\slshape\color[RGB]{128,0,0},   % 设置字符串格式
 showstringspaces=false,
 breaklines=true,
 language=c++
}

\begin{document}
    \maketitle
    \tableofcontents
    \section{实验目的}
    创建三个进程,分别为输入进程、处理进程及显示进程;用户在输入进程输入信息,处理进程会处理输入的信息,显示进程输出相应显示;
    三个进程之间通过不同方式进行通信,交换输入数据等等。
    \section{实验思想和方法}
    设置三个进程,分别为客户端进程(\emph{Client}),处理器进程(\emph{Deposit}),显示器进程(\emph{Display})。
    \begin{itemize}
        \item 客户端进程接收键盘输入,并\textbf{利用消息队列}将输入信息传递给显示器进程,\textbf{利用共享内存}将输入信息传递给处理器进程。
        \item 处理器进程\textbf{读取共享内存中的输入数据},将数字和字母分别写入不同的文件中,丢弃别的字符;之后\textbf{利用共享内存}给显示器进程发送信号,提醒其输出相应信息。
        \item 显示器进程\textbf{读取消息队列中的消息},将其显示在屏幕上;之后\textbf{读取共享内存接受信号},输出信息提醒用户其输入已经被处理。
        \item 在用户输入“quit”后关闭客户端进程、处理器进程和显示器进程。
    \end{itemize}
    \section{程序结构和算法}
\dirtree{%
    .1 lab4.
    .2 lab4\_pre.c\quad 封装函数、声明结构体.
    .2 lab4\_client.c\quad 接受输入,发送信息.
    .2 lab4\_process.c\quad 读入数据,处理数据. 
    .2 lab4\_display.c\quad 读入信息,显示信息. 
}

\begin{itemize}
    \item 创建4个.c文件,lab4\_pre.c用来封装函数,剩余三个对应三个进程,含有主函数。
    \item 在lab4\_pre.c中封装所有创建获得消息队列、删除消息队列、发送接收消息、创建获得共享内存、写入读取共享内存、删除共享内存的函数。
    \item 在lab4\_pre.c中定义消息的结构体,包含两个属性:消息类型和消息内容;
    \item 在lab4\_pre.c中定义共享内存的结构体,包含两个属性:是否被处理过的标识符和消息内容;
    \item 消息类型仅有一种,定义为INPUT宏=1;客户端进程向消息队列发送消息,显示器进程收取消息并显示;
    \item 写入共享内存时自动将标志符置0,读取时置1,如果处理器进程读取到输入为quit,则将标识符置2;客户端进程一旦发现此次输入的标识符已被置1,则输出提醒用户的信息;发现为2,则删除共享内存。
\end{itemize}
\section{运行结果}
运行结果如\reffig{fig_1}所示,从左到右分别是客户端进程、处理器进程和显示器进程,
可以发现,显示器进程正确地输出了所有输入的信息,并且提醒用户其输入数据已经被处理。\par 
\begin{figure}[htb]
    \centering
    \includegraphics[width = 16cm]{resources/done.png}
    \caption{运行结果}
    \label{fig_1}
\end{figure}

numbers.txt和letters.txt如\reffig{fig_2}所示,可见程序正确地将数字和字母分批写入到文件中。
\begin{figure}[htb]
    \centering
    \includegraphics[height = 3cm]{resources/numbers.png}
    \includegraphics[height = 3cm]{resources/letters.png}
    \caption{数字文件及字母文件}
    \label{fig_2}
\end{figure}

\clearpage
\section{问题分析}
\subsection{消息队列没有权限访问}
创建好的消息队列无法发送信息,通过\textbf{ipcs -p}检查发现其权限为0,发现原来在创建消息队列和共享内存时必须在IPC\_CREATE后或一个权限,否则无法访问。
\subsection{段错误}
最开始在消息队列的结构体中使用字符串指针而非字符数组,这时会出现段错误。我认为消息队列必须进行“深拷贝”,修改为字符数组后正常。
\subsection{消息队列删除}
在代码中调用msgctl(msg\_id,IPC\_RMID,NULL)删除消息队列后仍需要手动删除,即在终端中运行\textbf{ipcrm -q msg\_id}将其彻底删除,否则再次使用同样的key值创建消息队列时会报错。
\subsection{无法输出信息/无法写入信息到文件}
曾遇到无法使用printf("\%s",content)输出信息,亦无法使用fputc(char)写入字符到文件的问题,经查阅资料发现在写并发程序时需要即时使用fflush(stdout)/fflush(fp)清空缓存区,之后方可正常工作。
\subsection{比较消息队列和共享内存两种进程间通讯方式的利弊}
\begin{enumerate}[I.]
    \item 相比于消息队列,共享内存不用进行多次的拷贝,只需要各个进程访问同一个内存区就行;
    \item 相比于消息队列,共享内存不用在终端进行手动删除,可以完全由代码实现;
    \item 相比于消息队列,共享内存可以更方便地做到双方的交互,不用再次进行发收消息的动作;
    \item 相比于共享内存,消息队列实现了自然的读端互斥,仅有一个进程可以读取队列中的一个消息,读完后消息会自然删除;
\end{enumerate}\par 
\textbf{综上,在本实验的环境下,我认为共享内存优于消息队列。}\par 
此外,可以进一步探究消息队列实现多个读取段,以及共享内存设置互斥的办法,另外可以考虑用信号实现处理消息的反馈,更加符合逻辑。
\section{源代码}
\begin{itemize}
\item lab4\_pre.c
\begin{lstlisting}
#include<sys/types.h>
#include<sys/ipc.h>
#include<sys/msg.h>
#include<sys/shm.h>
#include<stdio.h>
#include<string.h>
#include<stdlib.h>
#define INPUT 1

struct msgItem
{
    long type;
    char text[1024];
}msgItem = {0,{0}};

struct shmItem
{
    int processed;
    char text[1024];
}shmItem = {0,{0}};

int initMsgQueue(int status){
    key_t key = ftok("./",0);
    if(key < 0){
        perror("failure in creating key");
    }
    int msg_id = msgget(key,status);
    if(msg_id < 0){
        perror("failure in creating message queue");
    }
    return msg_id;
}
int createMsgQueue(){
    return initMsgQueue(IPC_CREAT|0666);
}
int getMsgQueue(){
    return initMsgQueue(IPC_CREAT);
}
int destroyMsgQueue(int msg_id){
    if(msgctl(msg_id,IPC_RMID,NULL) < 0){
        perror("failure in destrpying message queue");
    }
    return 0;
}
int sendMsg(int msg_id,int type,char* message){
    struct msgItem  msgitem;
    msgitem.type = type;
    strcpy(msgitem.text,message);
    if(msgsnd(msg_id,(void*)&msgitem,sizeof(msgitem.text),0) < 0){
        perror("error in sending message");
        return -1;
    }
    return 0;
}

int recvMsg(int msg_id,int type,char * out){
    struct msgItem recvitem;
    int size = sizeof(recvitem.text);
    if(msgrcv(msg_id,(void*)&recvitem,size,type,0) < 0){
        perror("error in receiving message");
        return -1;
    }

    strcpy(out,recvitem.text);
    return 0;
}

int initShm(int status){
    key_t key = ftok("./",0);
    if(key < 0){
        perror("failure in creating key");
    }
    int msg_id = shmget(key,1024,status);
    if(msg_id < 0){
        perror("failure in creating shared memory");
    }
    //printf("%d",msg_id);
    return msg_id;
}

struct shmItem * createShm(){
    int shm_id = initShm(IPC_CREAT|0666);
    return (struct shmItem * )shmat(shm_id,0,0);
}

struct shmItem * getShm(){
    int shm_id = initShm(IPC_CREAT);
    return (struct shmItem * )shmat(shm_id,0,0);
}

int destroyShm(int shm_id,struct shmItem * shareMem){
    if(shmdt(shareMem) == -1){
        perror("Error in Unbounding Shared Memory");
    }
    if(shmctl(shm_id, IPC_RMID, 0) == -1){
        perror("Error in Destroying Shared Memory");
        return -1;
    }
};

int writeToShm(struct shmItem * shareMem,char * text){
    shareMem->processed = 0;
    strcat(shareMem->text,text);
    return 0;
}

int readFromShm(struct shmItem * shareMem,char out[]){
    shareMem->processed = 1;
    strcpy(out,shareMem->text);
    //puts(out);
    memset(shareMem->text,0,sizeof(out));
    return 0;
}
\end{lstlisting}
\item lab4\_client.c
\begin{lstlisting}
#include"lab4_pre.c"
int main(){
    int msg_id = createMsgQueue();
    struct shmItem * shared = createShm();
    //printf("%p",shared);
    printf("Process Client\n");
    char buf[100] = {0};
    while (1)
    {
        printf("type in message:\n");
        gets(buf);
        sendMsg(msg_id,(long)INPUT,buf);
        writeToShm(shared,buf);
        if(!strcmp("quit",buf)){
            return 0;
        }
    }
}
\end{lstlisting}
\item lsb4\_process.c
\begin{lstlisting}
#include "lab4_pre.c"
int main(){
    struct shmItem * shared = getShm();
    
    FILE * fp_num = fopen("numbers.txt","w+");
    FILE * fp_let = fopen("letters.txt","w+");
    printf("Process Deposit\n");
    while (1)
    {
        if(shared->processed){
            continue;
        }
        char out[100] = {};
        readFromShm(shared,out);
        if(!strcmp(out,"quit")){
            shared->processed = 2;
            return 0;
        }
        
        //printf("%s",out);
        //fflush(stdout);
        

        for(int i = 0;i < strlen(out);i++){
            int c = (int)out[i];
            //printf("%d",c);
            if((c<= 57 && c>=48)||c == 10){
                fputc(out[i],fp_num);
                fflush(fp_num);
            }
            else if((c<=122 && c>=65)||c==10){
                fputc(out[i],fp_let);
                fflush(fp_let);
            }
        }
        fputc('\n',fp_num);
        fputc('\n',fp_let);
        fflush(fp_num);
        fflush(fp_let);
    }
}
\end{lstlisting}
\item lab4\_display.c
\begin{lstlisting}
#include"lab4_pre.c"
int main(){
    int msg_id = getMsgQueue();
    int shm_id = initShm(IPC_CREAT);
    struct shmItem * shared = getShm();
    printf("Process Display\n");
    while (1)
    {
        char out[100] = {};
        recvMsg(msg_id,(long)INPUT,out);
        if(!strcmp("quit",out)){
            printf("message queue deleted(need to be deleted manually in shell)");
            destroyMsgQueue(msg_id);
            return 0;
        }
        printf("output:%s\n",out);
        while(!shared->processed){}
        if(shared->processed == 1){
            printf("Inputting Data has been Processed\n");
        }
        else
        {
            printf("Shared Memory Destroyed");
            destroyShm(shm_id,shared);
            return 0;
        }
    }
}
\end{lstlisting}
\end{itemize}
\end{document}