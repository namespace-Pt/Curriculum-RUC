\documentclass[UTF8]{ctexart}
\usepackage{dirtree}
\usepackage{listings}
\usepackage{xcolor}
\usepackage{graphicx}
\usepackage{enumerate}
\usepackage[a4paper]{geometry} 
\usepackage{amsmath,amsthm,mathtools}
\usepackage{mathtools}
\usepackage{diagbox}
\usepackage{multirow,makecell}
\usepackage{url}
\geometry{bottom=2cm}
\newcommand{\refe}[1]{Eq.\ref{#1}}
\newtheorem{theory}{Theory}[section]
\title{task5}
\author{张配天-2018202180}
\begin{document}
    \maketitle
    \section{7.5}
    \subsection{a}
    \begin{enumerate}[1.]
        \item 对比首次匹配,最差匹配\textbf{坏处}在于搜索过程更长需要更多时间,\textbf{好处}在于由于分配产生的内部碎片在之后会被更充分的利用;
        \item 对比下次匹配,最差匹配\textbf{坏处}在于搜索过程更长需要更多时间,\textbf{好处}在于由于分配产生的内存碎片在之后会被更充分的利用;
        \item 对比最佳匹配,最差匹配\textbf{坏处}在于对于单个进程浪费的存储空间很大,\textbf{好处}在于由于分配产生的内存碎片在之后会被更充分的利用;
    \end{enumerate}
    \subsection{b}
    $\Omega(n)$
    
    \section{7.7}
    如图
    \begin{figure}[htb]
        \centering
        \includegraphics[width=12cm]{resources/2020-06-04_13-34-13.png}
        \caption{分配表}
        \label{}
    \end{figure}
    \section{7.12}
    \subsection{a}
    $1G=2^{30}\, bytes$,因此$2G=2^{31}\, bytes$,即共需要31比特。
    \subsection{b}
    $1K=2^{10}\, bytes,1M=2^{11}\, bytes$
    \begin{itemize}
        \item 对于每个进程,都包含$\frac{256*2^{20}}{8*2^{10}} = 2^{15}$个页,所以每个进程需要15比特存储页号。
        \item 对于整个内存空间,都包含$\frac{2*2^{30}}{8*2^{10}} = 2^{18}$个页框,所以每个进程需要18个比特存储页框号。
    \end{itemize}
    \subsection{c}
    $2^{18}$
    \subsection{d}
    一个n+m位的地址,前n位表示该进程存的页号,后m位表示页内偏移量。\par 
    这里n=15,m=18。
    \section{7.14}
    \begin{enumerate}[a.]
        \item 物理地址:830+228 = 1058;
        \item $658>408$,因此产生段错误;
        \item 物理地址:770+776 = 1546;
        \item 物理地址:648+98=746;
        \item $240>110$,因此产生段错误;
    \end{enumerate}
\end{document}